\documentclass[11pt,twoside,a4paper,final]{report}

\usepackage{afterpage} 
\usepackage{amsmath}   
\usepackage{booktabs}  
\usepackage{enumerate} 
\usepackage{rotating}  
\usepackage{times}     
\usepackage{graphicx}  
\usepackage{natbib}    
% Using Hungarian Letters
\usepackage[magyar]{babel}
\usepackage[utf8]{inputenc} 
\usepackage[T1]{fontenc}

\raggedbottom
% Master Report Initial Details
\def\maintitle{Geo-Wiki}
\def\subtitle{Térkép alapú Wikipedia}
\def\authors{Nguyen Binh - ELTE IK}
\def\supervisors{Témavezető: Dr. Márton Mátyás}


% Start the document
\begin{document}


\newpage

% Page numbering, options are arabic (Arabic numerals), roman	(Lowercase Roman), Roman (Uppercase Roman)
% alph (lowercase letters) and Alph (Uppercase letters)
\pagenumbering{Roman}
\pagestyle{plain}

% Include table of contents
\tableofcontents

% Remove the % in front of the command below to include a list of tables
%\listoftables

% Remove the % in front of the command below to include a list of figures
%\listoffigures

\newpage
% Use arabic numbering throughout the text
\pagenumbering{arabic}

% Include an abstract of your report
\abstract{

Place your abstract text here

} 

% Start with the first chapter
\chapter{Bevezetés}
% With \label you can define a crossreference so that you can refer
% to this chapter later in the text by using Chapter~\ref{ch:background}
\label{ch:bevezetés}
	\section{Térképészet}
	\label{ch:térkép}
	
	\section{Informatika}
	\label{ch:informatika}

\chapter{Térképészet}
\label{sec:térképészet}
% Define the first section 
	\section{Térképszerkesztés, Térképtervezés}
	\label{sec:szerkesztés,tervezés}
	
	\section{Tematikus - Szak térképek ábrázolási módszerei}
	\label{sec:tematikus,szak,térkép,ábrázolás}

	\section{Tematikus - Szak térképek szerkesztése}
	\label{sec:tematikus,szak,szerkesztés}

\chapter{Wikipedia}
\label{ch:wiki,wikipedia}


\chapter{\ldots}
\label{ch:sample}

\ldots

   
% Insert the references
\bibliographystyle{plainnat}
\bibliography{library/studlib}

% Anything behind this command will be treated as an appendix
\appendix

% Insert the first appendix
\chapter{Appendix 1}
\label{app:appendix1}

Appendix text \ldots

% insert a second appendix
\chapter{Appendix 2}
\label{app:appendix2}

Appendix text \ldots

\end{document}
